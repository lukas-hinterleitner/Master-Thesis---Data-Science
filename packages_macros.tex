%here you can find some useful additional packages and macros

%Package to show algorithms
\usepackage{algorithm2e}

%package to work with glossaries
\usepackage[acronym]{glossaries}

%packages and setings to show colored code
\usepackage{listings}
\usepackage{xcolor}

\definecolor{codegreen}{rgb}{0,0.6,0}
\definecolor{codegray}{rgb}{0.5,0.5,0.5}
\definecolor{codepurple}{rgb}{0.58,0,0.82}
\definecolor{backcolour}{rgb}{0.95,0.95,0.92}

\lstdefinestyle{mystyle}{
    backgroundcolor=\color{backcolour},   
    commentstyle=\color{codegreen},
    keywordstyle=\color{magenta},
    numberstyle=\tiny\color{codegray},
    stringstyle=\color{codepurple},
    basicstyle=\ttfamily\footnotesize,
    breakatwhitespace=false,         
    breaklines=true,                 
    captionpos=b,                    
    keepspaces=true,                 
    numbers=left,                    
    numbersep=5pt,                  
    showspaces=false,                
    showstringspaces=false,
    showtabs=false,                  
    tabsize=2
}

\lstset{style=mystyle}

%package to allow comments/notes by you and other people
\usepackage[multiuser,marginclue,nomargin,inline,index]{fixme}
\definecolor{ttwgreen}{RGB}{75,135,73}
\fxusetheme{colorsig}
%\FXRegisterAuthor{tm}{atm}{\color{red}TM}
\FXRegisterAuthor{ttw}{attw}{\color{ttwgreen}TTW} % creates ttwnote, ttwwarning, ttwerror, ttwfatal commands
%\FXRegisterAuthor{mk}{amk}{\color{blue}MK}

%add your own packages or macros here
\usepackage{amssymb}
\usepackage{multirow}
\usepackage{caption}
\usepackage{subcaption}
\usepackage{float}

\SetKwComment{Comment}{/* }{ */}
\RestyleAlgo{ruled}

\newcommand*{\W}{\mathbf{W}}

\newcommand*{\lk}{(l,k)}

\newcommand*{\Wlk}{\W^{\lk}}
\newcommand*{\Wlonekone}{\W^{(l_1,k_1)}}
\newcommand*{\Wltwoktwo}{\W^{(l_2,k_2)}}
\newcommand*{\nWlk}{n_{\Wlk}}
\newcommand*{\dWlk}{d_{\Wlk}}
\newcommand*{\nWlkXdWlk}{\nWlk \times \dWlk}
\newcommand*{\RnWlkXdWlk}{\mathbb{R}^{\nWlkXdWlk}}

\newcommand*{\vWlk}{\vX[\Wlk]}

\newcommand*{\si}{s_i}
\newcommand*{\spi}{s_{p_i}}
\newcommand*{\smi}{s_{m_i}}

\newcommand*{\sj}{s_j}
\newcommand*{\spj}{s_{p_j}}
\newcommand*{\smj}{s_{m_j}}

\newcommand*{\vXsi}[1][\theta]{\mathbf{v}_{#1}(\si)}
\newcommand*{\vXspi}[1][\theta]{\mathbf{v}_{#1}(\spi)}
\newcommand*{\vXsmi}[1][\theta]{\mathbf{v}_{#1}(\smi)}

\newcommand*{\vXsj}[1][\theta]{\mathbf{v}_{#1}(\sj)}
\newcommand*{\vXspj}[1][\theta]{\mathbf{v}_{#1}(\spj)}
\newcommand*{\vXsmj}[1][\theta]{\mathbf{v}_{#1}(\smj)}

\newcommand*{\simcos}{\operatorname{sim}_{\cos}}

\newcommand{\Indic}{\mathbf{1}}

\newcommand{\BreakHeader}[2][c]{
\begin{tabular}[#1]{@{}c@{}}#2\end{tabular}
}